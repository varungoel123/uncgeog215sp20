\documentclass[11pt,]{article}
\usepackage[margin=1in]{geometry}
\newcommand*{\authorfont}{\fontfamily{phv}\selectfont}
\usepackage[]{mathpazo}
\usepackage{abstract}
\renewcommand{\abstractname}{}    % clear the title
\renewcommand{\absnamepos}{empty} % originally center
\newcommand{\blankline}{\quad\pagebreak[2]}

\providecommand{\tightlist}{%
  \setlength{\itemsep}{0pt}\setlength{\parskip}{0pt}} 
\usepackage{longtable,booktabs}

\usepackage{parskip}
\usepackage{titlesec}
\titlespacing\section{0pt}{12pt plus 4pt minus 2pt}{6pt plus 2pt minus 2pt}
\titlespacing\subsection{0pt}{12pt plus 4pt minus 2pt}{6pt plus 2pt minus 2pt}

\titleformat*{\subsubsection}{\normalsize\itshape}

\usepackage{titling}
\setlength{\droptitle}{-.25cm}

%\setlength{\parindent}{0pt}
%\setlength{\parskip}{6pt plus 2pt minus 1pt}
%\setlength{\emergencystretch}{3em}  % prevent overfull lines 

\usepackage[T1]{fontenc}
\usepackage[utf8]{inputenc}

\usepackage{fancyhdr}
\pagestyle{fancy}
\usepackage{lastpage}
\renewcommand{\headrulewidth}{0.3pt}
\renewcommand{\footrulewidth}{0.0pt} 
\lhead{}
\chead{}
\rhead{\footnotesize POSC 0000: A Class with an R Markdown Syllabus -- Fall 2016}
\lfoot{}
\cfoot{\small \thepage/\pageref*{LastPage}}
\rfoot{}

\fancypagestyle{firststyle}
{
\renewcommand{\headrulewidth}{0pt}%
   \fancyhf{}
   \fancyfoot[C]{\small \thepage/\pageref*{LastPage}}
}

%\def\labelitemi{--}
%\usepackage{enumitem}
%\setitemize[0]{leftmargin=25pt}
%\setenumerate[0]{leftmargin=25pt}




\makeatletter
\@ifpackageloaded{hyperref}{}{%
\ifxetex
  \usepackage[setpagesize=false, % page size defined by xetex
              unicode=false, % unicode breaks when used with xetex
              xetex]{hyperref}
\else
  \usepackage[unicode=true]{hyperref}
\fi
}
\@ifpackageloaded{color}{
    \PassOptionsToPackage{usenames,dvipsnames}{color}
}{%
    \usepackage[usenames,dvipsnames]{color}
}
\makeatother
\hypersetup{breaklinks=true,
            bookmarks=true,
            pdfauthor={ ()},
             pdfkeywords = {},  
            pdftitle={POSC 0000: A Class with an R Markdown Syllabus},
            colorlinks=true,
            citecolor=blue,
            urlcolor=blue,
            linkcolor=magenta,
            pdfborder={0 0 0}}
\urlstyle{same}  % don't use monospace font for urls


\setcounter{secnumdepth}{0}





\usepackage{setspace}

\title{POSC 0000: A Class with an R Markdown Syllabus}
\author{Steven V. Miller}
\date{Fall 2016}


\begin{document}  

		\maketitle
		
	
		\thispagestyle{firststyle}

%	\thispagestyle{empty}


	\noindent \begin{tabular*}{\textwidth}{ @{\extracolsep{\fill}} lr @{\extracolsep{\fill}}}


E-mail: \texttt{\href{mailto:svmille@clemson.edu}{\nolinkurl{svmille@clemson.edu}}} & Web: \href{http://svmiller.com/teaching}{\tt svmiller.com/teaching}\\
Office Hours: W 09:00-11:30 a.m.  &  Class Hours: TR 02:00-03:45 p.m.\\
Office: 230A Brackett Hall  & Class Room: \emph{online}\\
	&  \\
	\hline
	\end{tabular*}
	
\vspace{2mm}
	


\hypertarget{course-description}{%
\section{Course Description}\label{course-description}}

You'll learn stuff in this class, I hope. Lorem ipsum dolor sit amet,
consectetur adipiscing elit. Maecenas scelerisque elit sapien, eu
consequat dui blandit in. Vestibulum dignissim feugiat mauris, at
pretium turpis blandit nec. Aliquam porta scelerisque tortor, eget
imperdiet quam dapibus et. Sed ut sollicitudin orci, id elementum arcu.
Sed arcu quam, vestibulum molestie mattis sed, ultricies sed est.
Phasellus eu nunc et urna volutpat pharetra. Donec interdum ante vitae
odio malesuada blandit. Fusce at condimentum libero, eu elementum arcu.
Aenean posuere id lorem in varius. Sed bibendum neque pretium dolor
faucibus, in cursus ipsum suscipit. Lorem ipsum dolor sit amet,
consectetur adipiscing elit. Aliquam erat volutpat. Phasellus mollis
egestas risus, non maximus nisl euismod sit amet. Vestibulum laoreet et
urna vitae rutrum. Donec quis dui elit.

\hypertarget{course-objectives}{%
\section{Course Objectives}\label{course-objectives}}

\begin{enumerate}
\def\labelenumi{\arabic{enumi}.}
\item
  You'll learn this
\item
  And also that
\item
  Perhaps some of this too.
\end{enumerate}

\hypertarget{required-readings}{%
\section{Required Readings}\label{required-readings}}

\hypertarget{course-policy}{%
\section{Course Policy}\label{course-policy}}

I will detail the policy for this course below. Basically, don't cheat
and try to learn stuff. Don't be that guy.

\hypertarget{grading-policy} of your grade will be determined by a midterm during
  normal class hours.
\item
  \textbf{20\%} of your grade will be determined by a term paper that
  documents your appreciation of Foghat's ``Slow Ride'', the most
  important song ever written. ``Slow Ride'' is what Mozart wishes
  \emph{Don Giovanni} could have been.
\item
  \textbf{10\%} of your grade will be determined by your attendance and
  participation in class. Generally, ask questions and answer them.
\item
  \textbf{20\%} of your grade will be determined by a 20-page term paper
  on when exactly ``The Love Boat'' jumped the proverbial shark. You
  will address whether this shark-jumping can be attributed to Ted
  McGinley, the introduction of Jill Whelan as ``Vicki'', or some other
  cause.
\item
  \textbf{30\%} of your grade will be determined by a final exam.
\end{itemize}

\hypertarget{attendance-policy}{%
\subsection{Attendance Policy}\label{attendance-policy}}




\end{document}

\makeatletter
\def\@maketitle{%
  \newpage
%  \null
%  \vskip 2em%
%  \begin{center}%
  \let \footnote \thanks
    {\fontsize{18}{20}\selectfont\raggedright  \setlength{\parindent}{0pt} \@title \par}%
}
%\fi
\makeatother
